\documentclass[a4paper,12pt, french, twocolumn]{article}
\usepackage[french]{babel}

\PassOptionsToPackage{table}{xcolor}
\usepackage{onedown}
\setdefaults{colors=4A, bidlong=off, bidline =0}
\usepackage{tcolorbox}
\newcommand{\T}{\Cl}
\newcommand{\K}{\Di}
\newcommand{\C}{\He}
\renewcommand{\P}{\Sp}
\usepackage{graphicx}

\usepackage[Bjornstrup]{fncychap}

\usepackage{fontspec}
%\setmainfont{Universalis ADF Std}
%\setmainfont{liberation Sans}
\setmainfont{Caladea}
\newfontfamily{\myfont}{Caladea}[Ligatures = TeX]%[NFSSFamily=cmss]
\newfontfamily{\sansserif}{liberation Sans}[Ligatures = TeX]
\bidderfont{\mdseries\myfont}
\compassfont{\mdseries\sansserif}
\gamefont{\bfseries\sansserif}
\legendfont{\mdseries\sansserif}
\namefont{\mdseries\sansserif}
\otherfont{\bfseries\sansserif}


\definecolor{ashgrey}{rgb}{0.7, 0.75, 0.71}
\definecolor{apricot}{rgb}{0.98, 0.81, 0.69}
\definecolor{bananamania}{rgb}{0.98, 0.91, 0.71}

%\usepackage{fourier-otf}


\newcommand{\enchbox}[2]{
\begin{figure}[htbp]
\vspace{8pt}
\tcbox[left=0mm,right=0mm,top=0mm,bottom=0mm,boxsep=0mm,
toptitle=0.5mm,bottomtitle=0.5mm,title=\parenthese{#1}]{%fond vert
\arrayrulecolor{red!5!black}\renewcommand{\arraystretch}{1.2}%
\begin{tabular}{ccc}
 #2
\end{tabular}
}
\vspace{8pt}
\end{figure}
}

\rowcolors{1}{bananamania}{white}
\newcommand{\rb}{\rowcolor{bananamania} }
\newcommand{\rw}{\rowcolor{white} }
\newcommand{\ra}{\rowcolor{apricot} }

\tcbset{colframe=red!70!green,colback=white,colupper=orange!50!black,
fonttitle=\bfseries,nobeforeafter,center title}

\makeatletter
\renewcommand{\thesection}{\@arabic\c@section}
\makeatother

\usepackage{expl3}
\usepackage{xparse}

\ExplSyntaxOn

% Définir une nouvelle commande pour effectuer les substitutions
\NewDocumentCommand{\replacechars}{mmmmm}
 {
  \tl_set:Nn \l_tmpa_tl { #1 }
  \tl_replace_all:Nnn \l_tmpa_tl { #2 } { #3 }
  \tl_replace_all:Nnn \l_tmpa_tl { #4 } { #5 }
  \tl_use:N \l_tmpa_tl
 }

\ExplSyntaxOff

\newcommand{\parenthese}[1]{%
\replacechars{#1}{>}{)}{<}{(}%
}

\newcommand{\VoirChat}[1]
{}


\begin{document}



 {
 \centering

 {\huge\bfseries Deux sur Un\par}

 {\Large\itshape Forcing de manche\par}
}





\section*{Introduction}
Le but de cet article est de résumer ce que propose Alain Levy en ce qui concerne le Deux sur Un forcing de manche à des fins de mémorisations.
Par ailleurs, BBOAlert utilise les sources de ce document pour produire des alertes automatiques sur BBO. Ce qui est très pratique.

Il reprend les concepts intitulés Grizzly et Fairway.

Je reprend, sans les déformer et sans tenir compte de mes préférences personnelles, les développements proposés par Alain Levy.

%J'utilise la couleur abricot pour indiquer les séquences dont le traitement est très particulier qui nécessitent une mise au point ainsi qu'une mémorisation.

\section*{Réponses à l'ouverture}

Le système Y prévoir beaucoup de modifications par rapport au SEF2024.

En premier lieu, il utilise les JumpLimit. Ces enchères, redoutablement efficaces ont remplacé les Bergen aux USA depuis plus de vingt ans. Tous les sauts simples au niveau de 3 montent un unicolore limite de manche. Ces enchères ont l'avantage d'aller chercher une manche rapidement sans laisser au joueur numéro 4 l'opportunité d'intervenir. Il y a 5 séquences concernées 1\C--3\T, 1\C--3\K, 1\P--3\T, 1\P--3\K et 1\P--3C. Ces séquences remplacent leschangement de couleur avec répétition du SEF2024. Typiquement, la séquence 1\P--2\T--2\P--3\T du SEF2024 est remplacée par 1\P--3\T en système Y.

Le deuxième changement est la suppression du soutien à 3\NT qui entre dans le cadre des splinters. Le soutien fitté 2\NT devient bivalent. Il s'utilise avec les jeux limite de manche comme en SEF2024 mais aussi avec les mains banales de manche sans singleton de 12-14H. Ce système est proposé par Argine sur Funbridge depuis des années.

Le troisième changement est l'enchère \textit{semi-forcing }de 1\NT. Comme souvent, au bridge, les appellations sont trompeuses. Semi-forcing veut dire non forcing. En effet, les partisans du deux sur un forcing de manche se divisent en deux camp. Les partisans du 1\NT forcing, incluant quelques mains fortes, typiquement certains soutiens, et les partisans du 1\NT non forcing. En France, une majorité d'experts préfèrent jouer 1\NT non forcing. Le seul point à alerté ici, c'est la zone de points qui va jusque 11H.

\textit{Remarque : La réponse de 2\P sur l'ouverture de 1\C n'est pas traitée dans le tome 1 de l'ouvrage d'Alain Levy. Celui-ci indiquant que toutes les enchères fortes se trouveraient dans le tome 2, j'en ai inféré que cette enchère gardait sa signification traditionnelle d'enchère forte, sous toutes réserves. De toute façon, barrer les mineures avec les majeures serait par ailleurs peu subtil.}

\enchbox{Ouverture de 1\C}
{
1\P  & 6HL+ & 4+ cartes à \P \\
1\NT & 6-11H & semi-forcing  \\
2\T  & 12H+  & naturel ou fitté \\
2\K  & 12H+  & naturel \\
2\C & 8-10DH & 3 cartes à \C ou 4333\\
2\P & 16H+ & 6 cartes à \P \\
2\NT & 11-14DH & 3-4 cartes à \C \\
3\T & 9-11H  & 6 cartes à \T \\
3\K & 9-11H & 6 cartes à \K \\
3\C &  8-10DH & 4 cartes à \C \\
3\P & 7-10H & 5\C et chicane \T\K ou \P \\
3\NT & 7-10H & 5 atouts et singleton \P \\
4\T & 7-10H & 5 atouts et singleton \T \\
4\K & 7-10H & 5 atouts et singleton \K \\
4\C & & barrage \\
4\P && what else ? \\
4\NT && blackwood \\
}


\enchbox{Ouverture de 1\P}
{
1\NT & 6-11H & semi-forcing  \\
2\T  & 12H+  & naturel ou fitté \\
2\K  & 12H+  & 5+ cartes à \K \\
2\C  & 12H+ & naturel non fitté \\
2\P & 8-10DH &  3 cartes à \P ou 4333\\
2\NT & 11-14DH & 3-4 cartes à \P \\
3\T & 9-11H  & 6 cartes à \T \\
3\K & 9-11H & 6 cartes à \K \\
3\C &  9-11H & 6 cartes à \C \\
3\P & 8-10DH & 4 cartes à \P \\
3\NT & 7-10H & 5\P et chicane \T\K ou \C  \\
4\T & 7-10H & 5 atouts et singleton \T \\
4\K & 7-10H & 5 atouts et singleton \K \\
4\C & 7-10H & 5 atouts et singleton \C \\
4\P & & barrage
}

\section*{1\NT semi-forcing}

Il y a quelques séquences dont Alain levy ne précise pas la signification dans son ouvrage. Ce sont 1\P--1\NT--2\T--3\C et 1\P--1\NT--2\K--3\C. La logique d'Alain Levy conduirait à penser qu'il s'agit d'un 5-5 \T\C dans le premier cas et d'un 5-5 \K-\C dans le second, dans la zone forte (9)10-11H. Comme ce n'est pas explicité, je n'ai pas fait mention d'une telle possibilité dans les tableaux ci-contre.

Si on joue le système Y complet, la séquence 1\C--1\NT--2\T promet 5 cartes. En effet, avec une main 4522, jusqu'à 13H, on passe tranquillement sur 1\NT.
Et à partir de 14H, Alain Levy propose d'ouvrir de 1\NT les mains 4522. Sur le Texas et le Stayman, ça ne pose aucun problème particulier. Sur le Puppet, on dispose de la réponse gadget à 3\NT pour montrer ce 4522.
C'est pourquoi, dans le tableau, j'ai mis cette possibilité entre parenthèses.


La signification de l'enchère de 2\NT après les séquences 1\C-1\NT et 1\P-1\NT n'est pas explicite non plus dans le tome 1. Forcing ou non forcing, les deux sont possibles.

Il y a beaucoup de difficultés dans ces séquences.

Les redemandes du répondant à 3\T et 3\K sont toujours naturelles. Les enchères \textit{impossibles}\footnote{Les enchères dites impossibles servent à indiquer un soutien anormalement puissant dans la deuxième couleur de l'ouvreur} commencent à 3\P (éventuellement 3\C) sur ouverture de 1\P et à 3\C sur ouverture de 1\C.

D'autre part, le traitement de l'ouverture de 1\C est différent à plusieurs égards de celui de l'ouverture de 1\P. Après ouverture de 1\C, le répondant peut faire une enchère \textit{impossible} à 2\P. D'autre part, le traitement des jeux réguliers et des bicolores mineurs se fait à l'italienne lorsque l'ouvreur répète sa couleur. 1\C--1\NT--2\C--2\P et 1\P--1\NT--2\P--2\NT indiquent un jeu régulier de 10-11H. 1\C--1\NT--2\C--2\NT et 1\P--1\NT--2\P--3\T indiquent un bicolore mineur. Mnémotechniquement, il faut utiliser la logique italienne. Première enchère disponible avec 10-11H et deuxième enchère disponible avec le bicolore.

Dans la suite de son ouvrage, Alain Levy utilise beaucoup la logique italienne, notamment pour nommer certains singletons. La première enchère disponible indiquant le singleton le plus économique, et la deuxième indiquant le singleton plus cher. Cette logique est  peu pratiquée en France où on préfère la logique de substitution. On annonce explicitement le singleton et l'enchère impossible montre le singleton inexprimable. Ce qui pose problème quand il y a deux singletons tous les deux inexprimables, cas rarissime où on se rabat sur la logique italienne.




\enchbox{1\C--1\NT}
{
\rw 2\T && 4 cartes  ou\\
   && 14H+ (4522) 3523 ou 2533 \\
2\K && 4 cartes ou 14H+ 3532 \\
2\C && 6 cartes 11-15H \\
-> & 2\P & artificiel 10-11H régulier \\
\rb -> & 2\NT & bicolore mineur \\
-> & 3\T & 6-9H naturel \\
\rb -> & 3\K & 6-9H naturel \\
-> & 3\C & 11H singleton \C \\
}

\enchbox{1\C--1\NT--2\T}
{
2\K && naturel 6-9H \\
2\C && préférence \\
2\P & 10-11H & fit \T  \\
2\NT & 10-11H \\
3\T & 8-9H & naturel   \\
3\K & 6-9H & 6+ cartes \\
3\C & 10-11H & 5 cartes à \T Hx à \C \\
3\NT & 10-11H & 5 cartes à \T courte \C \\
}

\enchbox{1\C--1\NT--2\K}
{
2\C && préférence \\
2\P & 10-11H & fit \K  \\
2\NT & 10-11H \\
3\T & 6-9H & 6+ cartes   \\
3\K & 8-9H & naturel \\
3\C & 10-11H & 5 cartes à \K Hx à \C \\
3\NT & 10-11H & 5 cartes à \K courte \C \\
}

\enchbox{1\P--1\NT}
{
2\T && 4 cartes ou 14H+ et 3 cartes \\
2\K && 4 cartes ou 14H+ 5332 \\
2\C && 4 cartes \\
2\P && 6 cartes \\
}


\enchbox{1\P--1\NT--2\T}
{
2\K & 8-11H & texas \C facultatif \\
-> &2\C & au moins 2 cartes \\
\rw -> &2\P & singleton \C \\
-> & 3\C & 3 cartes et 14H+ \\
2\C & 6-9H & 6 cartes \\
2\P & & préférence \\
2\NT & 10-11H & \\
3\T & 8-9H & naturel \\
3\K & 6-9H & 6 cartes \\
3\C && hors système \\
3\P &10-11H& 5 cartes à \T Hx à \P \\
3\NT &10-11H & 5 cartes à \K courte \P\\
}

\enchbox{1\P--1\NT--2\K}
{
2\C & 6-9H & 5+ cartes \\
2\P & & préférence \\
2\NT & 10-11H & (2-5 cartes à \C)\\
-> & 3\C &14H+ et 3 cartes \\
3\T & 6-9H & 6 cartes \\
3\K & 8-9H & naturel \\
3\C && hors système \\
3\P &10-11H& 5 cartes à \T Hx à \P \\
3\NT &10-11H & 5 cartes à \K courte \P\\
}

\enchbox{1\P--1\NT--2\P}
{
2\NT && 10-11H ou longue \T \\
-> & 3\T & relai obligatoire \\
\rb ->-> & \Pass & pour les jouer \\
->-> & 3\K & force à \K \\
\rb ->-> & 3\C & force à \C \\
->-> & 3\P & limite de manche \\
3\T && bicolore mineur \\
3\K &6-9H & 6+ cartes \\
3\C & 6-9H & 6+ cartes \\
3\P & 11H & court à \P \\
4\P && pragmatique \\
}

\enchbox{1\P--1\NT--2\C}
{
2\P && préférence \\
2\NT &10-11H & \\
3\T & 6-9H & 6 cartes \\
3\K & 6-9H & 6 cartes \\
3\C & 8-11H & naturel \\
3\P &10-11H& 5 cartes à \C Hx à \P \\
3\NT &10-11H& 5 cartes à \C courte \P\\
4\T &10-11H& beau 5-5 \T-\C\\
4\K &10-11H& beau 5-5 \K-\C\\
}

\section*{Soutien différé au palier de 2}
%_1S--2C--2D--, 2S, 3 cartes à !S et 15DH+

Le changement de couleur à 2\T peut se faire avec 2 cartes à \T seulement.


\end{document}

